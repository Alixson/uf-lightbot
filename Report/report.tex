\documentclass[11pt, journal]{IEEEtran}

\title{Real-Time Facial Emotion Recognition}
\author{Robert~Kirchgessner,~\IEEEmembership{Graduate Student}
Tomasz Bacewicz~\IEEEmembership{Graduate Student}}

\begin{document}

\maketitle

\begin{abstract}
This paper explores a real-time system aimed at detecting facial emotions in a video stream.
The process of emotion recognition can be broken up into several stages: segmentation of facial regions within an image, extraction of facial features and categorization of features
into emotions~\cite{FacialExprRecogIntegratedApproach}. The first stage of processing identifies facial regions within a video frame. In recognizing the facial regions, three popular
methods are explored: skin-color face localization~\cite{FuzzyEmoModel}, AdaBoost and Automatically Generated 
Cascade Classifier (AM-CC)~\cite{RealTimeFaceDetection,ImprovedAdaboostFaceDetection}. The second stage of processing identifies features which can
be used as classifiers in recognizing facial expressions. Multiple feature detection methods are explored: directional templates for evaluating 
monochromatic intensity distributions, edge-like blob maps~\cite{FacialFeatHumanRobo} and multi-resolution analysis~\cite{RealTimeFeatureExtraction}.
Finally, an artificial neural network is used in order to classify the identified features into Ekman's six basic emotions~\cite{FacialExprRecogIntegratedApproach}. The neural network is trained using images taken from available databases such as the CMU/VASC face database.
\end{abstract}

\section{Introduction}
\section{Related Work}
\section{Description}
\section{Evaluation}
\section{Related Work}
\section{Summary and Conclusions}

\newpage 
\bibliography{bibliography}
\bibliographystyle{IEEEtran}

\end{document}
